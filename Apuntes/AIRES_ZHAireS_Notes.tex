\documentclass[a4paper,12pt]{article}
\usepackage[spanish]{babel}         
\usepackage[utf8]{inputenc}           
\usepackage[T1]{fontenc}
\usepackage{graphicx}
\usepackage{color}
\usepackage{tikz}
\usepackage{anysize}
\usepackage{multicol}
\usepackage{bm}
\usepackage{textcomp}
\usepackage{eurosym}
\usepackage{amsthm}
\usepackage{amsmath}
\usepackage{amsfonts}
\usepackage{amssymb}
\usepackage{lineno}
\usepackage{epstopdf}
\usepackage{fancyhdr}
\usepackage{subfigure}
\usepackage{float}
\usepackage[square, numbers]{natbib}
\usepackage{longtable}
\usepackage{multirow}
\usepackage{array}
\usepackage{tabularx}
\usepackage[linkcolor=blue,colorlinks,]{hyperref}
\usepackage[all]{hypcap}
%\usepackage{lipsum}
\usepackage{mathdots}
\usepackage{yhmath}
\usepackage{cancel}
%\usepackage{siunitx}
\usepackage{gensymb}
\usepackage{booktabs}

\usetikzlibrary{fadings}
\usetikzlibrary{patterns}

\marginsize{1.5cm}{1.5cm}{2cm}{2cm} % MÁRGENES: Izq, Der, Sup, Inf.
\parindent=0mm                        % Sangría por defecto. 
\parskip=3mm                          % Espacio entre párrafos por defecto.
\renewcommand{\baselinestretch}{1}    % Interlineado.

\renewcommand{\headrulewidth}{0pt}
\renewcommand{\footrulewidth}{0pt}
\renewcommand{\spanishtablename}{Tabla} %Denominacion para tablas
\renewcommand{\spanishabstractname}{} %Denominacion para el abstract
\renewcommand{\spanishrefname}{} %Denominacion para las referencias
\renewcommand{\spanishfigurename}{Figura} %Denominacion para imagenes
\renewcommand{\spanishchaptername}{} %Denominacion para capitulos
\renewcommand{\spanishcontentsname}{Índice} %Denominacion para el indice

\fancyhead[LE,RO]{} %Cabecera a la izquierda  
\fancyhead[RE,LO]{} %Cabecera a la derecha
\fancyfoot[LE,RO]{} %Abajo a la izquierda
\fancyfoot[RE,LO]{} %Abajo a la derecha    %Creo que al reves en clase article
\fancyfoot[C]{} %En el centro
\pagestyle{fancy}

\numberwithin{equation}{section}
\numberwithin{figure}{section}
\setcounter{tocdepth}{1}
%opening
\title{\textbf{AIRES, ZHAireS y RASPASS: Cuentas y plantillas}}
\author{Sergio Cabana Freire}
\date{\today} 


\begin{document}   
	
	\maketitle
	\begin{abstract}
		Voy a recoger aquí las cuentas que vaya haciendo sobre la geometría de los programas. También incluiré algunas plantillas para input files y las normas que se me ocurran para nombrar ficheros de manera eficiente. Lo que sigue está pensado para AIRES 19.04.08 (22/Oct/2021) con la extensión ZHAireS 1.0.30a (30/Jun/2022), que incluye RASPASS\footnote{ \textit{Raspass is an Aires Special Primary for Atmospheric Skimming Showers}. Primera sigla recursiva observada en la naturaleza.}. Las diferencias con versiones anteriores estarán en la geometría (aunque los cálculos podrán adaptarse, simplemente fijando algunos parámetros que en RASPASS son libres) y en alguna tabla o instrucción IDL nueva, nada más.
		
		
		\vspace{10mm}
		Referencias importantes:
		
		\href{http://aires.fisica.unlp.edu.ar/}{Página web de AIRES / ZHAireS. Documentación actualizada}
		
		\href{https://indico.cern.ch/event/826366/contributions/4881703/attachments/2459764/4218361/Tueros-%20ARENA%2020222%20-%20RASPASS%20Showers.pdf}{Presentación de RASPASS. Matías Tueros, ARENA 2022}
		
		\href{https://github.com/SergioCabana/PhD_thesis}{Repositorio GitHub (con algunas plantillas y códigos que pueden ser útiles)}
	\end{abstract}
	\tableofcontents
	\newpage
	
	\section{Geometría en RASPASS}
	\tikzset{every picture/.style={line width=0.75pt}} %set default line width to 0.75pt        
	\begin{figure}[H]
	\centering
	\begin{tikzpicture}[x=0.75pt,y=0.75pt,yscale=-1,xscale=1]

		%uncomment if require: \path (0,458); %set diagram left start at 0, and has height of 458
		
		%Straight Lines [id:da24054111542845868] 
		\draw    (357.8,23) -- (357.3,441) ;
		\draw [shift={(357.8,20)}, rotate = 90.07] [fill={rgb, 255:red, 0; green, 0; blue, 0 }  ][line width=0.08]  [draw opacity=0] (8.93,-4.29) -- (0,0) -- (8.93,4.29) -- cycle    ;
		%Straight Lines [id:da7710588561097498] 
		\draw    (40,202) -- (357.8,143) ;
		\draw [shift={(357.8,143)}, rotate = 349.48] [color={rgb, 255:red, 0; green, 0; blue, 0 }  ][fill={rgb, 255:red, 0; green, 0; blue, 0 }  ][line width=0.75]      (0, 0) circle [x radius= 3.35, y radius= 3.35]   ;
		\draw [shift={(40,202)}, rotate = 349.48] [color={rgb, 255:red, 0; green, 0; blue, 0 }  ][fill={rgb, 255:red, 0; green, 0; blue, 0 }  ][line width=0.75]      (0, 0) circle [x radius= 3.35, y radius= 3.35]   ;
		%Straight Lines [id:da8825568085983706] 
		\draw  [dash pattern={on 4.5pt off 4.5pt}]  (357.8,143) -- (574.85,101.56) ;
		\draw [shift={(577.8,101)}, rotate = 169.19] [fill={rgb, 255:red, 0; green, 0; blue, 0 }  ][line width=0.08]  [draw opacity=0] (8.93,-4.29) -- (0,0) -- (8.93,4.29) -- cycle    ;
		%Straight Lines [id:da09435750281004718] 
		\draw    (357.3,181) -- (490.8,181.98) ;
		\draw [shift={(493.8,182)}, rotate = 180.42] [fill={rgb, 255:red, 0; green, 0; blue, 0 }  ][line width=0.08]  [draw opacity=0] (8.93,-4.29) -- (0,0) -- (8.93,4.29) -- cycle    ;
		%Straight Lines [id:da2786415192647693] 
		\draw    (40,202) -- (357.3,441) ;
		%Straight Lines [id:da05933265632010665] 
		\draw  [dash pattern={on 4.5pt off 4.5pt}]  (9.79,206.84) -- (40,202) ;
		\draw [shift={(10.77,206.68)}, rotate = 170.9] [fill={rgb, 255:red, 0; green, 0; blue, 0 }  ][line width=0.08]  [draw opacity=0] (8.93,-4.29) -- (0,0) -- (8.93,4.29) -- cycle    ;
		%Curve Lines [id:da9168852208815295] 
		\draw [color={rgb, 255:red, 248; green, 231; blue, 28 }  ,draw opacity=1 ][line width=1.5]    (333.8,147) .. controls (332.8,131) and (336.8,122) .. (357.8,116) ;
		%Curve Lines [id:da23814962464976852] 
		\draw    (344,146) .. controls (343.8,153) and (348.8,157) .. (356.8,156) ;
		%Straight Lines [id:da8178069675474764] 
		\draw [color={rgb, 255:red, 208; green, 2; blue, 27 }  ,draw opacity=1 ][line width=1.5]    (30.95,137.46) -- (344.85,79.54) ;
		\draw [shift={(347.8,79)}, rotate = 169.55] [color={rgb, 255:red, 208; green, 2; blue, 27 }  ,draw opacity=1 ][line width=1.5]    (14.21,-6.37) .. controls (9.04,-2.99) and (4.3,-0.87) .. (0,0) .. controls (4.3,0.87) and (9.04,2.99) .. (14.21,6.37)   ;
		\draw [shift={(28,138)}, rotate = 349.55] [color={rgb, 255:red, 208; green, 2; blue, 27 }  ,draw opacity=1 ][line width=1.5]    (14.21,-6.37) .. controls (9.04,-2.99) and (4.3,-0.87) .. (0,0) .. controls (4.3,0.87) and (9.04,2.99) .. (14.21,6.37)   ;
		%Straight Lines [id:da10918348040808445] 
		\draw [color={rgb, 255:red, 189; green, 16; blue, 224 }  ,draw opacity=1 ][line width=1.5]    (368.07,178) -- (368.73,148) ;
		\draw [shift={(368.8,145)}, rotate = 91.27] [color={rgb, 255:red, 189; green, 16; blue, 224 }  ,draw opacity=1 ][line width=1.5]    (14.21,-6.37) .. controls (9.04,-2.99) and (4.3,-0.87) .. (0,0) .. controls (4.3,0.87) and (9.04,2.99) .. (14.21,6.37)   ;
		\draw [shift={(368,181)}, rotate = 271.27] [color={rgb, 255:red, 189; green, 16; blue, 224 }  ,draw opacity=1 ][line width=1.5]    (14.21,-6.37) .. controls (9.04,-2.99) and (4.3,-0.87) .. (0,0) .. controls (4.3,0.87) and (9.04,2.99) .. (14.21,6.37)   ;
		%Straight Lines [id:da33357063373554574] 
		\draw    (29.62,218.17) -- (119.18,282.83) ;
		\draw [shift={(120.8,284)}, rotate = 215.83] [color={rgb, 255:red, 0; green, 0; blue, 0 }  ][line width=0.75]    (10.93,-4.9) .. controls (6.95,-2.3) and (3.31,-0.67) .. (0,0) .. controls (3.31,0.67) and (6.95,2.3) .. (10.93,4.9)   ;
		\draw [shift={(28,217)}, rotate = 35.83] [color={rgb, 255:red, 0; green, 0; blue, 0 }  ][line width=0.75]    (10.93,-4.9) .. controls (6.95,-2.3) and (3.31,-0.67) .. (0,0) .. controls (3.31,0.67) and (6.95,2.3) .. (10.93,4.9)   ;
		%Straight Lines [id:da5898839119963639] 
		\draw    (128.62,288.18) -- (348.18,447.62) ;
		\draw [shift={(349.8,448.8)}, rotate = 215.99] [color={rgb, 255:red, 0; green, 0; blue, 0 }  ][line width=0.75]    (10.93,-4.9) .. controls (6.95,-2.3) and (3.31,-0.67) .. (0,0) .. controls (3.31,0.67) and (6.95,2.3) .. (10.93,4.9)   ;
		\draw [shift={(127,287)}, rotate = 35.99] [color={rgb, 255:red, 0; green, 0; blue, 0 }  ][line width=0.75]    (10.93,-4.9) .. controls (6.95,-2.3) and (3.31,-0.67) .. (0,0) .. controls (3.31,0.67) and (6.95,2.3) .. (10.93,4.9)   ;
		%Straight Lines [id:da739942628197686] 
		\draw    (371.97,187) -- (367.83,436.8) ;
		\draw [shift={(367.8,438.8)}, rotate = 270.95] [color={rgb, 255:red, 0; green, 0; blue, 0 }  ][line width=0.75]    (10.93,-4.9) .. controls (6.95,-2.3) and (3.31,-0.67) .. (0,0) .. controls (3.31,0.67) and (6.95,2.3) .. (10.93,4.9)   ;
		\draw [shift={(372,185)}, rotate = 90.95] [color={rgb, 255:red, 0; green, 0; blue, 0 }  ][line width=0.75]    (10.93,-4.9) .. controls (6.95,-2.3) and (3.31,-0.67) .. (0,0) .. controls (3.31,0.67) and (6.95,2.3) .. (10.93,4.9)   ;
		%Straight Lines [id:da7919092562367724] 
		\draw    (629.38,212.4) -- (577.42,263.6) ;
		\draw [shift={(576,265)}, rotate = 315.42] [color={rgb, 255:red, 0; green, 0; blue, 0 }  ][line width=0.75]    (10.93,-4.9) .. controls (6.95,-2.3) and (3.31,-0.67) .. (0,0) .. controls (3.31,0.67) and (6.95,2.3) .. (10.93,4.9)   ;
		\draw [shift={(630.8,211)}, rotate = 135.42] [color={rgb, 255:red, 0; green, 0; blue, 0 }  ][line width=0.75]    (10.93,-4.9) .. controls (6.95,-2.3) and (3.31,-0.67) .. (0,0) .. controls (3.31,0.67) and (6.95,2.3) .. (10.93,4.9)   ;
		%Curve Lines [id:da3011792723205482] 
		\draw [line width=3]    (108,302) .. controls (234.8,143) and (477.8,138) .. (605.8,300) ;
		%Curve Lines [id:da11287344757921924] 
		\draw [line width=3]    (15,300) .. controls (121.8,84) and (469.8,13) .. (660.8,240) ;
		
		% Text Node
		\draw (359.8,17) node [anchor=south west] [inner sep=0.75pt]   [align=left] {$\displaystyle z$};
		% Text Node
		\draw (495.8,182) node [anchor=west] [inner sep=0.75pt]   [align=left] {$\displaystyle x$};
		% Text Node
		\draw (169,80) node [anchor=north west][inner sep=0.75pt]  [font=\large,color={rgb, 255:red, 208; green, 2; blue, 27 }  ,opacity=1 ] [align=left] {$\displaystyle d$};
		% Text Node
		\draw (378,151) node [anchor=north west][inner sep=0.75pt]  [font=\large,color={rgb, 255:red, 144; green, 19; blue, 254 }  ,opacity=1 ] [align=left] {$\displaystyle h$};
		% Text Node
		\draw (317,113) node [anchor=north west][inner sep=0.75pt]  [font=\large,color={rgb, 255:red, 248; green, 231; blue, 28 }  ,opacity=1 ] [align=left] {$\displaystyle \theta $};
		% Text Node
		\draw (379,290) node [anchor=north west][inner sep=0.75pt]   [align=left] {$\displaystyle R_{T}$};
		% Text Node
		\draw (202,363) node [anchor=north west][inner sep=0.75pt]   [align=left] {$\displaystyle R_{T}$};
		% Text Node
		\draw (63,252) node [anchor=north west][inner sep=0.75pt]   [align=left] {$\displaystyle v$};
		% Text Node
		\draw (30,176) node [anchor=north west][inner sep=0.75pt]   [align=left] {IP};
		% Text Node
		\draw (471,78) node [anchor=north west][inner sep=0.75pt]   [align=left] {Eje de la cascada};
		% Text Node
		\draw (605.4,241) node [anchor=north west][inner sep=0.75pt]   [align=left] {$\displaystyle l_{atm}$};
		% Text Node
		\draw (482,265) node [anchor=north west][inner sep=0.75pt]   [align=left] {Tierra};
		
	\end{tikzpicture}
		\caption{Esquema de la geometría en RASPASS}
		\label{fig11}
		\end{figure}
	
	Tenemos tres parámetros para determinar la geometría en RASPASS:
	\begin{itemize}
		\item RASPASSDistance, $d$. Distancia a lo largo del eje desde el punto de inyección (IP) hasta la intersección con la vertical del observador.
		\item RASPASSHeight,  $h$. Altura a la que pasa el eje de la cascada sobre la vertical del observador.
		\item PrimaryZenAngle, $\theta$. Ángulo entre el eje de la cascada y la vertical del observador.
	\end{itemize}
Esta geometría puede reducirse a casos más sencillos:
\begin{itemize}
	\item Cascadas hacia abajo (normales): $h = 0$, $\theta<90^\circ$.
	\item Cascadas hacia arriba: $h = 0$, $d \leq 0$, $\theta >90^\circ$ (Inyección a nivel del suelo o por encima)
\end{itemize}
Todo lo que sigue es trigonometría sencilla\footnote{$\sin(\pi-x)=\sin{x}$ ; $\cos(\pi-x)=-\cos{x}$}:
\begin{equation}
	(R_T+v)^2=(R_T+h)^2+d^2+2d(R_T+h)\cos\theta
	\label{ec11}
\end{equation}
\newpage
\subsection{Altura del punto de inyección}
Despejando de \eqref{ec11}:
\begin{equation}
	v = \sqrt{(R_T+h)^2+d^2+2d(R_T+h)\cos\theta}-R_T
	\label{ec12}
\end{equation}
\subsection{Valor de $d$ en función de $v$ (y demás parámetros RASPASS)}
La ecuación \eqref{ec11} es una cuadrática en d:
\begin{equation}
	d=-(R_T+h)\cos\theta+\sqrt{(R_T+h)^2\cos^2\theta-\left[h^2-v^2+2R_T(h-v)\right]}
	\label{ec13}
\end{equation}
\subsection{Altura en la atmósfera, $H$, en función de distancia recorrida a lo largo del eje desde IP}
\begin{figure}[H]
	\centering
\begin{tikzpicture}[x=0.75pt,y=0.75pt,yscale=-1,xscale=1]
	%uncomment if require: \path (0,300); %set diagram left start at 0, and has height of 300
	
	%Straight Lines [id:da5554117546302768] 
	\draw    (90,80.78) -- (473.23,268) ;
	%Straight Lines [id:da38790146210321685] 
	\draw    (90,80.78) -- (520.8,44) ;
	\draw [shift={(90,80.78)}, rotate = 355.12] [color={rgb, 255:red, 0; green, 0; blue, 0 }  ][fill={rgb, 255:red, 0; green, 0; blue, 0 }  ][line width=0.75]      (0, 0) circle [x radius= 3.35, y radius= 3.35]   ;
	%Straight Lines [id:da42930679641114566] 
	\draw    (473.23,268) -- (520.8,44) ;
	%Straight Lines [id:da02718269904522752] 
	\draw    (269,66) -- (473.23,268) ;
	%Straight Lines [id:da2933478969179921] 
	\draw    (83.99,51.82) -- (507.81,13.18) ;
	\draw [shift={(509.8,13)}, rotate = 174.79] [color={rgb, 255:red, 0; green, 0; blue, 0 }  ][line width=0.75]    (10.93,-3.29) .. controls (6.95,-1.4) and (3.31,-0.3) .. (0,0) .. controls (3.31,0.3) and (6.95,1.4) .. (10.93,3.29)   ;
	\draw [shift={(82,52)}, rotate = 354.79] [color={rgb, 255:red, 0; green, 0; blue, 0 }  ][line width=0.75]    (10.93,-3.29) .. controls (6.95,-1.4) and (3.31,-0.3) .. (0,0) .. controls (3.31,0.3) and (6.95,1.4) .. (10.93,3.29)   ;
	%Straight Lines [id:da43176526833890105] 
	\draw    (191,58) -- (256.81,51.21) ;
	\draw [shift={(258.8,51)}, rotate = 174.11] [color={rgb, 255:red, 0; green, 0; blue, 0 }  ][line width=0.75]    (10.93,-3.29) .. controls (6.95,-1.4) and (3.31,-0.3) .. (0,0) .. controls (3.31,0.3) and (6.95,1.4) .. (10.93,3.29)   ;
	%Straight Lines [id:da6310470056534143] 
	\draw    (163,61) -- (90.79,67.81) ;
	\draw [shift={(88.8,68)}, rotate = 354.61] [color={rgb, 255:red, 0; green, 0; blue, 0 }  ][line width=0.75]    (10.93,-3.29) .. controls (6.95,-1.4) and (3.31,-0.3) .. (0,0) .. controls (3.31,0.3) and (6.95,1.4) .. (10.93,3.29)   ;
	
	% Text Node
	\draw (355.33,112.11) node [anchor=north west][inner sep=0.75pt]  [rotate=-44.67] [align=left] {$\displaystyle R_{T} +H$};
	% Text Node
	\draw (254.99,166.91) node [anchor=north west][inner sep=0.75pt]  [rotate=-27.01] [align=left] {$\displaystyle R_{T} +v$};
	% Text Node
	\draw (500.31,176.61) node [anchor=north west][inner sep=0.75pt]  [rotate=-281.5] [align=left] {$\displaystyle R_{T} +h$};
	% Text Node
	\draw (466,59) node [anchor=north west][inner sep=0.75pt]   [align=left] {$\displaystyle \pi -\theta $};
	% Text Node
	\draw (129,83) node [anchor=north west][inner sep=0.75pt]   [align=left] {$\displaystyle \alpha $};
	% Text Node
	\draw (173,49) node [anchor=north west][inner sep=0.75pt]   [align=left] {$\displaystyle L$};
	% Text Node
	\draw (284,9) node [anchor=north west][inner sep=0.75pt]   [align=left] {$\displaystyle d$};
	% Text Node
	\draw (67,83) node [anchor=north west][inner sep=0.75pt]   [align=left] {IP};
	% Text Node
	\draw (293,67) node [anchor=north west][inner sep=0.75pt]   [align=left] {$\displaystyle \alpha ^{*}$};
	
	
\end{tikzpicture}

\caption{Detalle del triángulo en la Fig. \ref{fig11}.}
\label{fig12}
\end{figure}
Aplicando el teorema del coseno:
\begin{equation}
	(R_T+H)^2=(R_T+v)^2+L^2-2L(R_T+v)\cos{\alpha}
	\label{ec14}
\end{equation}
\begin{equation}
	H=\sqrt{(R_T+v)^2+L^2-2L(R_T+v)\cos{\alpha}}-R_T\label{ec15}
\end{equation}
$v$ puede obtenerse de los inputs de RASPASS \eqref{ec12}. Falta calcular $\cos\alpha$. Aplicamos el teorema del seno y elevamos al cuadrado:
\begin{equation}
	\frac{(R_T+h)^2}{1-\cos^2{\alpha}}=\frac{(R_T+v)^2}{\sin^2(\pi-\theta)}=\frac{(R_T+v)^2}{\sin^2\theta}\label{ec16}
\end{equation}
\begin{equation}
	\cos{\alpha}=\sqrt{1-\left(\frac{R_T+h}{R_T+v}\right)^2\sin^2\theta}\label{ec17}
\end{equation}
Las ecuaciones \eqref{ec15} y \eqref{ec17} forman el resultado.
\subsection{Materia atravesada en función de distancia recorrida a lo largo del eje desde IP}
La cantidad de materia atravesada es:
\begin{equation}
	X(L)\,\left[\mathrm{g/cm^2}\right]=\int_{0}^L\rho(H(l))dl\label{ec18}
\end{equation}
donde $dl$ es el elemento de línea a lo largo del eje. Es fácil ver que, al avanzar a lo largo del eje una distancia $dl$, la altura cambia en:
\begin{equation}
	dH = \pm dl \cos{\alpha(H)^*} = \mathrm{sgn}\left(\frac{\delta H}{\delta l}\right) dl \cos{\alpha(H)^*}\label{ec19}
\end{equation}
donde $\alpha^*$ es el ángulo cenital \textit{local}, i.e., entre una distancia recorrida $L$ y $L+dl$ (Fig. \ref{fig12}). El signo depende de si la altura aumenta o disminuye según avanzamos. Planteando el teorema del seno como en \eqref{ec16}, es inmediato llegar a que\footnote{ Ojo, el cambio de variable va con el \textit{valor absoluto} del determinante jacobiano en la integral. El hecho de meter el signo se debe a que $X$ debe ser una magnitud estrictamente positiva.}:
\begin{equation}
	X(L) = \int_{v}^{H(L)}\mathrm{sgn}\left(\frac{\delta H'}{\delta l}\right)\rho(H')\frac{dH'}{\sqrt{1-\left(\frac{R_T+h}{R_T+H'}\right)^2\sin^2\theta}}\label{ec110}
\end{equation}
Puede que haya que tener cuidado con el hecho de que $H(L)$ decrece y luego vuelve a crecer en cascadas estratosféricas, y el signo cambia. En ese caso:
\begin{equation}
	X(L) = \int_{H_{min}}^{v}\rho(H')\frac{H'}{\cos\alpha(H')^*}+\int_{H_{min}}^{H(L)}\rho(H')\frac{dH'}{\cos\alpha(H')^*}\label{ec111}
\end{equation}

\section{Normas para nombrar \textit{Tasks} en AIRES}
Los archivos .inp y outputs de AIRES pueden dar lugar a bastantes dolores de cabeza, a la hora de identificar qué archivos son los correspondientes a qué simulación y viceversa. Se me han ocurrido unas normas para poner nombres a los \textit{Tasks} en AIRES de manera sistemática. La idea es que viendo el nombre de un archivo podamos identificar inmediatamente la simulación en cuestión y sus parámetros claves, además de poder hacer de manera más sencilla gráficas y análisis posteriores (ya que teniendo unas normas para los nombres de archivos, podremos simplemente escribir un código que haga el trabajo de buscar los ficheros de datos necesarios para la gráfica que queremos, por ejemplo).

El formato típico para un \textit{task} será:

\begin{verbatim}
	(SpecialParticleCode_)Primary_Energy_TrajectoryParams_ExtraComments
\end{verbatim}
Es decir, indicaremos:
\begin{itemize}
	\item Indicador de partícula especial: Si hemos usado una primaria introducida mediante la instrucción \verb|AddSpecialParticle| y un código \textit{externo} a AIRES. Los códigos más habituales (en mi caso) serían:
	\verb|uprimary_...| , \verb|RAS_...| (para RASPASS)
	\item Partícula primaria: Siempre con dos caracteres o más.
\begin{table}[H]
	\centering
	\begin{tabular}{|l|l|}
		\hline
		No válido &\verb|p_|... , \verb|e_|...                                              \\ \hline
		Válido    & \verb|prot_|...   ,    \verb|elec_|...   ,   \verb|gamma_|...   ,    \verb|iron_|... \\ \hline
	\end{tabular}
\end{table}

La norma general para las partículas comunes será usar \verb|prot|, \verb|elec|, \verb|neut|. Para las demás partículas y núcleos, siempre el nombre completo: \verb|muon|, \verb|pion|, \verb|kaon|, \verb|lead|, ...

\item Energía de la partícula primaria: En $\mathrm{eV}$ con formato exponencial. Por ejemplo, ...\verb|_1e18eV_|..., ...\verb|_5.7e15eV_|... 
\item Trayectoria: Depende del tipo de partícula, ya que añadir primarios especiales normalmente implica cambiar los parámetros que definen la trayectoria. En cualquier caso, los ángulos se indicarán en grados y las distancias en $\mathrm{km}$ (con las unidades explícitas). Todos los parámetros se indicarán con  al menos dos cifras numéricas y se separarán con \verb|_|. En los casos habituales:

\begin{table}[H]
	\centering
	\begin{tabular}{c|c|}
		\cline{2-2}
		\multicolumn{1}{l|}{}                & Norma                                                                     \\ \hline
		\multicolumn{1}{|c|}{AIRES (normal)} & ...\verb|_InjectionAltitude_Zenith_Azimuth_|...                                \\ \hline
		\multicolumn{1}{|c|}{uprimary}       & ...\verb|_ALTITUDE_Zenith_Azimuth_|...                                         \\ \hline
		\multicolumn{1}{|c|}{RASPASS}        & ...\verb|_RASPASSDistance_RASPASSHeight_RASPASSTimeShift_Zenith_Azimuth_|... \\ \hline
	\end{tabular}
\end{table}

\begin{table}[H]
	\centering
	\begin{tabular}{c|c|}
		\cline{2-2}
		& Ejemplo                                  \\ \hline
		\multicolumn{1}{|c|}{AIRES (normal)} & ...\verb|_100km_65deg_00deg_|...              \\ \hline
		\multicolumn{1}{|c|}{uprimary}       & ...\verb|_00km_95deg_00deg_|...                 \\ \hline
		\multicolumn{1}{|c|}{RASPASS}        & ...\verb|1500km_35km_00ns_93.5deg_00deg_|... \\ \hline
	\end{tabular}
\end{table}
Las cascadas hacia arriba tendrán un ángulo cenital $>90^\circ$ por convenio. Por defecto, suponemos que el azimut es magnético (si fuera geográfico, añadimos una \verb|G| a las unidades.)
Como norma general para otros casos, deben indicarse todos los parámetros que describan las trayectorias.
\item Comentarios extra: Para incluir cualquier información relevante para identificar la simulación y sus condiciones. Algunos ejemplos pueden ser:
\begin{enumerate}
	\item \textit{La simulación incluye cálculo de emisión electromagnética (en tiempo, frecuencias o ambos)}: ...\verb|_ZHS(tfb)|
	\item \textit{Índice de refracción fijado}: ...\verb|nconst|
	\item \textit{Se ha fijado un campo magnético constante}: ...\verb|fixB|
	\item \textit{Se promedian 5 cascadas}: ...\verb|_05show|
	\item \textit{La cascada se simula en Malargüe}: ...\verb|_SiteMalargue|
	\item \textit{Se utiliza el modelo de atmósfera 3}: ...\verb|_Atmos3|
	\item \textit{El suelo está a $35\,\mathrm{m}$ a.s.l.}: ...\verb|_grd35m|
\end{enumerate}
Como recomendación, está bien usar más de dos cifras para cualquier magnitud numérica.
\end{itemize} 
Algunos ejemplos pueden ser:
\begin{itemize}
	\item Cascada iniciada por un protón de $100\,\mathrm{TeV}$, inyectado a $100\,\mathrm{km}$ de altura con ángulos $\theta=65^\circ$, $\varphi=0^\circ$, con el suelo a $100\,\mathrm{m}$ a.s.l.
	\begin{verbatim}
		prot_1e17eV_100km_65deg_00deg_grd100m
	\end{verbatim}
	\item Cascada hacia arriba iniciada por un electrón de $1,5\,\mathrm{EeV}$, interaccionando a $1,\mathrm{km}$ del suelo, con ángulos $\theta = 95^\circ$, $\varphi=65^\circ$ (geográfico), calculando emisión en radio en dominio temporal con índice de refracción constante:
	\begin{verbatim}
		uprimary_elec_1e18eV_01km_95deg_65degG_ZHSt_nconst
	\end{verbatim}
\item Cascada estratosférica iniciada por un protón (RASPASS) de $1\,\mathrm{EeV}$, a ángulo cenital $93,25^\circ$ y azimut $0^\circ$. La trayectoria cruza la vertical del observador a $35\,\mathrm{km}$ de altura (RASPASSHeight) y el primario se inyecta a $2000\,\mathrm{km}$ de dicho cruce (RASPASSDistance). No se impone TimeShift, se calcula emisión en radio en ambos dominios con un campo magnético fijado.
\begin{verbatim}
	RAS_prot_1e18eV_2000km_35km_00ns_93.25deg_00deg_ZHSb_fixB
\end{verbatim}
\end{itemize}

\end{document}
